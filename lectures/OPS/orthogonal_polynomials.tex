\documentclass{amsart}

\usepackage{amsmath,amsthm,amssymb,bm}
\usepackage{hyperref}
\usepackage{a4wide}
\usepackage{cleveref}
% \usepackage{refcheck}
\usepackage{graphicx,color}
\usepackage{tikz}
\numberwithin{equation}{section}

\newtheorem{thm}{Theorem}[section]
\newtheorem{lem}[thm]{Lemma}
\newtheorem{prop}[thm]{Proposition}
\newtheorem{cor}[thm]{Corollary}
\theoremstyle{definition}
\newtheorem{exam}[thm]{Example}
\newtheorem{defn}[thm]{Definition}
\newtheorem{conj}[thm]{Conjecture}
\newtheorem{question}[thm]{Question}
\newtheorem{problem}[thm]{Problem}
\newtheorem{remark}[thm]{Remark}
\newtheorem*{note}{Note}

% DO NOT DELETE THIS COMMENT!!! MACROS BELOW:
\newcommand\NN{\mathbb{N}}
\newcommand\QQ{\mathbb{Q}}
\newcommand{\CC}{\mathbb{C}}
\newcommand{\ZZ}{\mathbb{Z}}
\newcommand{\RR}{\mathbb{R}}
\newcommand\LL{\mathcal{L}}

\newcommand\Mot{\operatorname{Mot}}
\newcommand{\Dyck}{\operatorname{Dyck}}

\newcommand\Par{\operatorname{Par}}
\newcommand\RPP{\operatorname{RPP}}
\newcommand\SSYT{\operatorname{SSYT}}
\newcommand\SYT{\operatorname{SYT}}

\newcommand\wt{\operatorname{wt}}

\renewcommand\vec[1]{\mathbf{#1}}
\newcommand\flr[1]{\left\lfloor #1\right\rfloor}
\newcommand\Qbinom[3]{\genfrac{[}{]}{0pt}{}{#1}{#2}_{#3}}
\newcommand\qbinom[2]{\Qbinom{#1}{#2}{q}}

\newcommand\hyper[5]{{}_{#1}F_{#2} \left(#3;#4;#5\right)}
\newcommand\qhyper[5]{{}_{#1}\phi_{#2} \left(#3;#4;#5\right)}
\newcommand\Hyper[5]{{}_{#1}F_{#2} \left( \left.
    \begin{matrix}
      #3\\
      #4\\
    \end{matrix}
    \:\right|\: #5
    \right)}
\newcommand\qHyper[5]{{}_{#1}\phi_{#2} \left(
    \begin{matrix}
      #3\\
      #4\\
    \end{matrix}
    ; #5
    \right)}

\newcommand\comment[1]{\textcolor{blue}{\bf #1}}

%%%%%%%%%%%%%%%%%%%%%%%%%%%%%%%%%%%%%%%%%%%%%%%%%%%%%%%%%%%%%%%%

\title{Combinatorics of orthogonal polynomials}
\author{Jang Soo Kim}
% \thanks{The author was supported by NRF grants \#2022R1A2C101100911 and \#2016R1A5A1008055.} 
% \address{Department of Mathematics,
% Sungkyunkwan University (SKKU), Suwon, Gyeonggi-do 16419, South Korea}
% \email{jangsookim@skku.edu}
\date{\today}

\begin{document}

% \begin{abstract}
% \end{abstract}


\maketitle
\tableofcontents

%%%%%%%%%%%%%%%%%%%%%%%%%%%%%%%%%%%%%%%%%%%%%%%%%%%%%%%%%%%%%%%%
% start here

\section{Introduction to the lectures}

Orthogonal polynomials are classical objects arising from the study of
continued fractions. Due to the long history of orthogonal
polynomials, they have now become important objects of study in many
areas: classical analysis and PDE, mathematical physics, probability,
random matrix theory, and combinatorics.

The combinatorial study of orthogonal polynomials was pioneered by
Flajolet and Viennot in 1980s. In these lectures we will learn
fascinating combinatorial properties of orthogonal polynomials.

We will first study basic properties of orthogonal polynomials based
on Chihara's book, Chapter~1 \cite{Chihara}. We will then focus on the
combinatorial approach of orthogonal polynomials, which will be based
on Viennot's lecture notes \cite{ViennotLN}. We will also cover more
recent developments in the combinatorics of orthogonal polynomials
such as their connections with ASEP, staircase tableaux, lecture hall
partitions, and orthogonal polynomials of type \( R_1 \).

The prerequisites of this course are Calculus 1, Linear Algebra, and
Discrete Mathematics.


\section{Elementary Theory of Orthogonal Polynomials}

In this section we will cover the first chapter of Chihara's book
\cite{Chihara}.

\subsection{Introduction}

Since
\[
  2\cos m\theta \cos n\theta  = \cos(m+n)\theta + \cos(m-n)\theta,
\]
for nonnegative integers \( m \) and \( n \), we have
\begin{equation}\label{eq:coscos=0}
  \int_0^\pi \cos m\theta \cos n\theta d\theta = 0, \qquad m\ne n.
\end{equation}
In this situation we say that \( \cos m\theta \) and
\( \cos n\theta \) are orthogonal over the interval \( (0,\pi) \).

Note that \( \cos n\theta \) is a polynomial in \( \cos\theta \) of
degree \( n \). So we can write \( \cos n\theta = T_n(\cos\theta) \)
for a polynomial \( T_n(x) \) of degree \( x \).

By the change of variable \( x=\cos \theta \), \eqref{eq:coscos=0} can
be rewritten as
\[
  \int_{-1}^1 T_m(x)T_n(x) (1-x^2)^{-1/2} dx = 0, \qquad m\ne n.
\]

The polynomials \( T_n(x) \), \( n\ge0 \), are called the
\emph{Tchebyshev polynomials of the first kind}.
The first few polynomials are:
\begin{align*}
 T_0(x) &= 1, \\
 T_1(x) &= \cos\theta = x, \\
 T_2(x) &= \cos2\theta = 2\cos^2\theta-1 = 2x^2-1, \\
 T_3(x) &= 4x^3-3x.
\end{align*}

Recall that in an inner product space \( V \) with inner product
\( \langle \cdot,\cdot \rangle \), a set of vectors
\( v_1,\dots,v_n \) are said to be orthogonal if
\( \langle v_i,v_j \rangle = 0 \) for all \( i\ne j \). In this sense
the Tchebyshev polynomials \( T_n(x) \) are orthogonal, where
\( V = \RR[x] \) is the space of polynomials with real coefficients
with the inner product given by
\[
  \langle f(x), g(x) \rangle = \int_{-1}^1 f(x)g(x) (1-x^2)^{-1/2} dx.
\]
We say that \( T_n(x) \) are \emph{orthogonal polynomials} with
respect to the \emph{weight function} \( (1-x^2)^{-1/2} \) on the
interval \( (-1,1) \).

\begin{defn}\label{def:OPS1}
  Suppose that \( w(x) \) is a nonnegative and integrable function on
  \( (a,b) \) with \( \int_a^b w(x)dx >0 \) and
  \( \int_a^b x^n dx < \infty \) for all \( n\ge0 \). A sequence of
  polynomials \( \{P_n(x)\}_{n\ge0} \) is called an \emph{orthogonal
    polynomial sequence (OPS)} with respect to the \emph{weight
    function} (or \emph{measure}) \( w(x) \) on \( (a,b) \) if the
  following conditions hold:
  \begin{enumerate}
  \item \( \deg P_n(x) = n \), for \( n\ge0 \),
  \item \( \int_a^b P_m(x)P_n(x) w(x)dx = 0 \) for \( m\ne n \).
  \end{enumerate}
\end{defn}

There is another way to define orthogonal polynomials without using
the weight function. For a polynomial \( f(x) \), if we define
\[
  \LL(f(x)) = \int_a^b f(x) w(x)dx,
\]
then \( \LL(f(x)) \) is completely determined by the \emph{moments}
\( \mu_n = \int_a^b x^n w(x)dx \). So, if we are only interested in
polynomials, then we can define a linear functional \( \LL \) using a
moment sequence \( \mu_0,\mu_1,\dots \). Not every sequence
\( \mu_0,\mu_1,\dots \) gives rise to an OPS, though. We will see
later a criterion for a sequence to be a moment sequence.

\begin{defn}\label{def:OPS2}
  Let \( \LL \) be the linear functional defined on the space of
  polynomials in \( x \). A sequence of polynomials
  \( \{P_n(x)\}_{n\ge0} \) is called an \emph{orthogonal polynomial
    sequence (OPS)} with respect to \( \LL \) if the following
  conditions hold:
  \begin{enumerate}
  \item \( \deg P_n(x) = n \), \( n\ge0 \),
  \item \( \LL(P_m(x)^2) \ne 0 \) for \( m\ge0 \),
  \item \( \LL(P_m(x)P_n(x))  = 0 \) for \( m\ne n \).
  \end{enumerate}
\end{defn}

Note that the second condition above was not necessary in
\Cref{def:OPS1} because it follows from the facts that \( w(x) \) is
nonnegative and \( \int_a^b w(x)dx >0 \).

\begin{remark}
  The moments of the Tchebyshev polynomials are
  \[
    \mu_{2n} = \int_{-1}^1 x^{2n} (1-x^2)^{-1/2} dx
    = \frac{\pi}{2^{2n}} \binom{2n}{n}, \qquad
    \mu_{2n+1} = 0.
  \]
  This suggests that there could be some interesting combinatorics
  behind the scene. We will later find a combinatorial way to
  understand this situation.
\end{remark}


\begin{exam}[Charlier polynomials]
  The \emph{Charlier polynomials} \( P_n(x) \) are defined by
  \[
    P_n(x) = \sum_{k=0}^{n} \binom{x}{k} \frac{(-a)^{n-k}}{(n-k)!},
  \]
  where \( \binom{x}{k} = x(x-1)\cdots(x-k+1)/k! \).
  We will find a different type of orthogonality for \( P_n(x) \).

  The generating function for \( P_n(x) \) is
  \[
    G(x,w) = \sum_{n\ge0} P_n(x) w^n
    = \sum_{n\ge0} \left( \sum_{k=0}^{n} \binom{x}{k} \frac{(-a)^{n-k}}{(n-k)!} \right) w^n
    = \sum_{n\ge0} \binom{x}{n} w^n \sum_{n\ge 0} \frac{(-a)^m}{m!} w^m,
  \]
  which means
  \[
    G(x,w) =  e^{-aw}(1+w)^x .
  \]
  Thus
  \[
    a^x G(x,v) G(x,w) = e^{-a(v+w)} \left( a(1+v)(1+w) \right)^x .
  \]

  We have
  \[
    \sum_{k\ge 0} \frac{a^k G(k,v)G(k,w)}{k!}
    = \sum_{k\ge 0} \frac{e^{-a(v+w)} \left( a(1+v)(1+w) \right)^k}{k!} 
    = e^{-a(v+w)} e^{a(1+v)(1+w)} = e^ae^{avw}.
  \]
  Thus
  \begin{equation}\label{eq:G(k,v)G(k,w)}
    \sum_{k\ge 0} \frac{a^k G(k,v)G(k,w)}{k!}
    = \sum_{n\ge 0} \frac{e^a (avw)^n}{n!}.
  \end{equation}

On the other hand
\begin{align}
  \notag
    \sum_{k\ge 0} \frac{a^k G(k,v)G(k,w)}{k!}
    &= \sum_{k\ge 0} \frac{a^k}{k!} \sum_{m,n\ge0} P_m(k) P_n(k) v^m w^n\\
  \label{eq:PPvw}
    &= \sum_{m,n\ge0} \left( \sum_{k\ge 0}  P_m(k) P_n(k) \frac{a^k}{k!} \right)v^m w^n.
\end{align}
Comparing the coefficients of \( v^mw^n \) in \eqref{eq:G(k,v)G(k,w)} and \eqref{eq:PPvw} we obtain
\begin{equation}\label{eq:charlier-orthogonality}
  \sum_{k\ge 0}  P_m(k) P_n(k) \frac{a^k}{k!} = \frac{e^a a^n}{n!} \delta_{n,m}.
\end{equation}
Therefore, if we define a linear functional \( \LL \) by
\[
 \LL(x^n)  = \sum_{k\ge 0} k^n \frac{a^k}{k!},
\]
then \( P_n(x) \) are orthogonal polynomials with respect to \( \LL \).

Note that we describe the orthogonality of \( P_n(x) \) using only the
linear functional \( \LL \) without referring to any weight function.
However, we can also find a weight function in this case. Let
\( \psi(x) \) be the step function with a jump at \( k=0,1,2,\ldots \) of
magnitude \( a^k/k! \).
Then the linear functional \( \LL \) can be written as the following
Riemann--Stieltjes integral
\[
  \LL(f(x)) = \int_{-\infty}^\infty f(x) d \psi(x).
\]
\end{exam}

\begin{remark}
We can also prove \eqref{eq:charlier-orthogonality} in a combinatorial way.
To do this, note that
\begin{align*}
  \sum_{k\ge 0}  P_m(k) P_n(k) \frac{a^k}{k!}
  &=  \sum_{k\ge 0} \sum_{i=0}^m \binom{k}{i} \frac{(-a)^{m-i}}{(m-i)!}
  \sum_{j=0}^n \binom{k}{j} \frac{(-a)^{n-j}}{(n-j)!} \frac{a^k}{k!}\\
  &=  \sum_{k\ge 0} \sum_{i=0}^m \sum_{j=0}^n
    \binom{k}{m-i} \frac{(-a)^{i}}{i!}
   \binom{k}{n-j} \frac{(-a)^{j}}{j!} \frac{a^k}{k!}\\
  &= \sum_{N\ge0} \frac{a^N}{N!} \sum_{i+j+k=N}
  (-1)^{i+j} \frac{N!}{i!j!k!} \binom{k}{m-i} \binom{k}{n-j}.
\end{align*}

For a fixed \( N \),
\[
  \sum_{i+j+k=N} (-1)^{i+j} \binom{N}{i,j,k} \binom{k}{m-i} \binom{k}{n-j}
  = \sum_{(A,B,C)\in X} (-1)^{|B\setminus A| + |C\setminus A|},
\]
where \( X \) is the set of triples \( (A,B,C) \) such that
\( A\cup B\cup C = \{ 1,\dots,N \} \), \( |A|=k, |B|=m, |C|=n \),
\( (B\cap C)\setminus A = \emptyset \). If \( B\ne C \), we can find a
sign-reversing involution, namely toggling the smallest integer in
regions \( 1 \) and \( 2 \) or in regions \( 3 \) and \( 4 \) in
Figure~\ref{fig:image1}.
\begin{figure}
  \centering
  \includegraphics[scale=.1]{./figures/image1.jpeg}
  \caption{The triple \( (A,B,C) \).}
  \label{fig:image1}
\end{figure}
Thus we have
\[
  \sum_{(A,B,C)\in X} (-1)^{|B\setminus A| + |C\setminus A|}
  = \sum_{(A,B,C)\in X: B=C} (-1)^{|B\setminus A| + |C\setminus A|}
  = \delta_{m,n} \binom{N}{n}.
\]
This implies
\[
  \sum_{k\ge 0}  P_m(k) P_n(k) \frac{a^k}{k!}
  = \delta_{m,n} \sum_{N\ge0} \frac{a^N}{N!} \binom{N}{n}
  = \frac{e^a a^n}{n!} \delta_{n,m}.
\]
\end{remark}



\begin{remark}
  In the theory of orthogonal polynomials, finding an explicit weight
  function is an important problem. However, in these lectures, we
  will not pursue in this direction and we will be mostly satisfied
  with \Cref{def:OPS2}.
\end{remark}


\subsection{The moment functional and orthogonality}

We will consider the space \( \CC[x] \) polynomials with complex
coefficients. A \emph{linear function} on \( \CC[x] \) is a map
\( \LL:\CC[x] \to \CC \) such that
\( \LL(af(x)+bg(x)) = a\LL(f(x)) + b\LL(g(x)) \) for all
\( f(x), g(x)\in \CC[x] \) and \( a,b\in \CC \).

\begin{defn}
  Let \( \{\mu_{n}\}_{n\ge0} \) be a sequence of complex numbers. Let
  \( \LL \) be the linear functional on the space of polynomials
  defined by \( \LL(x^n) = \mu_n \), \( n\ge0 \). In this case we say
  that \( \LL \) is the \emph{moment functional} determined by the
  \emph{moment sequence} \( \{\mu_n\} \), and \( \mu_n \) is called
  the \emph{\( n \)th moment}.
\end{defn}

We recall the definition of orthogonal polynomials.

\begin{defn}
  Let \( \LL \) be the linear functional defined on the space of
  polynomials in \( x \). A sequence of polynomials
  \( \{P_n(x)\}_{n\ge0} \) is called an \emph{orthogonal polynomial
    sequence (OPS)} with respect to \( \LL \) if the following
  conditions hold:
  \begin{enumerate}
  \item \( \deg P_n(x) = n \), \( n\ge0 \),
  \item \( \LL(P_m(x)P_n(x))  = K_n \delta_{m,n} \), for some \( K_n\ne 0 \).
  \end{enumerate}
\end{defn}

We say that \( P_n(x) \) are \emph{orthonormal} if
\( \LL(P_m(x)P_n(x)) = \delta_{m,n} \).


\begin{thm}\label{thm:orth-equiv}
  Let \( \{P_n(x) \} \) be a sequence of polynomials and let \( \LL \) be
  a moment sequence. The following are equivalent:
  \begin{enumerate}
  \item \( \{P_n(x) \} \) is an OPS with respect to \( \LL \);
  \item \( \LL(\pi(x) P_n(x)) = 0 \) if \( \deg \pi(x) < n \) and
    \( \LL(\pi(x) P_n(x)) \ne 0 \) if \( \deg \pi(x) = n \);
  \item \( \LL(x^m P_n(x)) = K_n \delta_{m,n} \), \( 0\le m\le n \), for some \( K_n\ne 0 \).
  \end{enumerate}
\end{thm}
\begin{proof}
\( (1) \Rightarrow (2) \):
Suppose that \( \deg \pi(x) \le n \).
Since \( \{P_n(x) \} \) is a basis of \( \CC[x] \), we can write
\[
  \pi(x) = c_0 + c_1 P_1(x) + \cdots + c_n P_n(x).
\]
Then
\[
  \LL(\pi(x) P_n(x)) = \sum_{k=0}^n \LL \left( c_k P_k(x) P_n(x) \right)
  = c_n \LL(P_n(x)^2),
\]
which is zero if \( \deg \pi(x) <n \) and nonzero if \( \deg \pi(x) =n \).

\( (2) \Rightarrow (3) \): Trivial.
\( (2) \Rightarrow (3) \): Trivial.
\end{proof}

\begin{thm}\label{thm:orth-coeff}
  Suppose that \( \{ P_n(x) \}_{n\ge 0} \) be an OPS with respect to \( \LL \).
  Then for any polynomial \( \pi(x) \) of degree \( n \),
\[
  \pi(x) = \sum_{k=0}^n  c_k P_k(x), \qquad
  c_k = \frac{\LL(\pi(x)P_k(x))}{\LL(P_k(x)^2)}.
\]
\end{thm}
\begin{proof}
  Clearly, we can write
  \[
    \pi(x) = \sum_{k=0}^n c_k P_k(x),
  \]
  for some \( c_k \). Multiplying \( P_j(x) \) both sides
  and taking \( \LL \), we get
  \[
    \LL(\pi(x) P_j(x)) = \sum_{k=0}^n \LL \left( c_k P_k(x) P_j(x) \right)
    = c_j \LL(P_j(x)^2).
  \]
  Dividing both sides by \( \LL(P_j(x)^2) \), we obtain the theorem.
\end{proof}

\begin{thm}
  Suppose that \( \{ P_n(x) \}_{n\ge 0} \) be an OPS with respect to
  \( \LL \). Then \( P_n(x) \) is uniquely determined by \( \LL \) up
  to a nonzero factor. More precisely, if \( \{ Q_n(x) \}_{n\ge 0} \)
  is an OPS with respect to \( \LL \), then there are constants
  \( c_n\ne0 \) such that \( Q_n(x) = c_n P_n(x) \) for all
  \( n\ge0 \).
\end{thm}
\begin{proof}
  Let us write \(Q_n(x) = \sum_{k=0}^n c_k P_k(x) \). Then by
  \Cref{thm:orth-coeff}, \( c_k = \LL(Q_n(x)P_k(x))/\LL(P_k(x)^2) \).
  But by \Cref{thm:orth-equiv}, \( \LL(Q_n(x)P_k(x)) = 0 \) unless
  \( k=n \). Thus \( Q_n(x) = c_n P_n(x) \).
\end{proof}


Note that if \( \{ P_n(x) \}_{n\ge 0} \) is an OPS for \( \LL \), then
so is \( \{ c_nP_n(x) \}_{n\ge 0} \) for any \( c_n\ne 0 \). Therefore
there is a unique monic OPS, which is obtained by dividing each
\( P_n(x) \) by its leading coefficient. Note also that there is a
unique orthonormal OPS as well given by
\( p_n(x) = P_n(x)/\LL(P_n(x)^2) \). In summary we have the following
corollary.

\begin{cor}\label{cor:OPS-unique}
  Suppose that \( \LL \) is a moment sequence such that there is an
  OPS for \( \LL \). Let \( K_n \), \( n\ge0 \), be a sequence of
  nonzero numbers. 
  \begin{enumerate}
  \item There is a unique monic OPS \( \{ P_n(x) \}_{n\ge 0} \) for \( \LL \).
  \item There is a unique orthonormal OPS \( \{ P_n(x) \}_{n\ge 0} \) for \( \LL \).
  \item There is a unique OPS \( \{ P_n(x) \}_{n\ge 0} \) for \( \LL \)
    such that the leading coefficient of \( P_n(x) \) is \( K_n \).
  \item There is a unique OPS \( \{ P_n(x) \}_{n\ge 0} \) for \( \LL \)
    such that \( \LL(x^nP_n(x)) = K_n \).
  \end{enumerate}
\end{cor}


Clearly, if \( \{ P_n(x) \}_{n\ge 0} \) is an OPS for \( \LL \), then
it is also an OPS for \( \LL' \) given by
\( \LL'(f(x)) = c \LL(f(x)) \) for some \( c\ne 0 \). Therefore, by
dividing the linear functional by the value \( \LL(1) \), we may
assume that \( \LL(1)=1 \).


\subsection{Existence of OPS}

The main question in this section is: for what moment functional
\( \LL \) does there exist an OPS? To answer this question we need the
following definition.

\begin{defn}
  The \emph{Hankel determinant} of a moment sequence \( \{\mu_n\} \)
  is defined by
  \[
    \Delta_n = \det(\mu_{i+j})_{i,j=0}^n
    = \begin{vmatrix}
        \mu_0 & \mu_1 & \cdots & \mu_n\\
        \mu_1 & \mu_2 & \cdots & \mu_{n+1}\\
        \vdots & \vdots & \ddots & \vdots\\
        \mu_n & \mu_{n+1} & \cdots & \mu_{2n}
      \end{vmatrix} .
  \]
\end{defn}

\begin{thm}
  Let \( \LL \) be a moment functional with moment sequence
  \( \{\mu_n\} \).
  Then there is an OPS for \( \LL \) if and only if
  \( \Delta_n\ne 0 \) for all \( n\ge0 \).
\end{thm}

\begin{proof}
  Fix a sequence \( \{K_n\} \) of nonzero real numbers \( K_n \). By
  \Cref{cor:OPS-unique}, if there is an OPS
  \( \{ P_n(x) \}_{n\ge 0} \) for \( \LL \), it is uniquely determined
  by the condition \( \LL(x^n P_n(x)) = K_n \), \( n\ge0 \). In other
  words, using \Cref{thm:orth-equiv}, there is an OPS for \( \LL \) if
  and only if there is a unique sequence \( \{ P_n(x) \}_{n\ge 0} \)
  of polynomials such that
  \begin{equation}\label{eq:1}
    \LL(x^mP_n(x)) = K_n \delta_{m,n}, \qquad 0\le m\le n.
  \end{equation}
  
  Now let \( P_n(x) = \sum_{k=0}^{n} c_{n,k} x^k \). Multiplying both
  sides by \( x^m \) and taking \( \LL \), we get
  \[
    \LL(x^mP_n(x)) = \sum_{k=0}^{n} c_{n,k} \mu_{n+k}.
  \]
  Thus (\ref{eq:1}) can be written as the matrix equation
  \begin{equation}\label{eq:2}
   \begin{pmatrix}
     \mu_0 & \mu_1 & \cdots & \mu_n\\
     \mu_1 & \mu_2 & \cdots & \mu_{n+1}\\
     \vdots & \vdots & \ddots & \vdots\\
     \mu_n & \mu_{n+1} & \cdots & \mu_{2n}
   \end{pmatrix}
\begin{pmatrix}
c_{n,0} \\ c_{n,1} \\ \vdots \\ c_{n,n}
\end{pmatrix} 
= \begin{pmatrix}
0 \\ \vdots \\ 0 \\ K_n
\end{pmatrix}. 
  \end{equation}
  Then the uniqueness of the polynomials \( P_n(x) \) satisfying
  (\ref{eq:1}) is equivalent to the uniqueness of the solution of the
  matrix equation (\ref{eq:2}) in \( c_{n,0},c_{n,1},\dots,c_{n,n} \).
  In order for (\ref{eq:2}) to have a unique solution, the Hankel
  determinant \( \Delta_n \) must be nonzero for all \( n\ge0 \). This
  proves the theorem.
\end{proof}

Note that by solving \eqref{eq:2} using Cramer's rule, we have
\( c_{n,n} = K_n \Delta_{n-1}/\Delta_n \), which is the leading
coefficient of \( P_n(x) \).

% \subsection{The fundamental recurrence formula}


% \subsection{Zeros}


% \subsection{Gauss quadrature (maybe skip)}


% \subsection{Kernel polynomials (skip)}


% \subsection{Symmetric moment functionals (skip)}


% \subsection{Certain related recurrence relations (skip)}




% \section{Formal power series and generating functions}

% \section{Viennot theory}


% \section{LGV lemma}


% \section{Basics on orthogonal polynomials}


% \section{Basics on combinatorics}


% \section{linearization coefficients}

% \section{Other topics}

% \subsection{Hypergeometric series}

% \subsection{Connection with random matrices}

% \subsection{ASEP}





\bibliographystyle{abbrv}
\bibliography{/Users/jangsookim/Library/CloudStorage/Dropbox/newbiemacs/nbm-user-settings/references/ref.bib}

\end{document}
