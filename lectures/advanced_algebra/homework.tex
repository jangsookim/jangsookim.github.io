\documentclass{amsart}

\usepackage{amsmath,amsthm,amssymb,bm}
\usepackage{hyperref}
\usepackage{a4wide}
\usepackage{cleveref}
% \usepackage{refcheck}
\usepackage{graphicx,color}
\usepackage{tikz}
\numberwithin{equation}{section}

\newtheorem{thm}{Theorem}[section]
\newtheorem{lem}[thm]{Lemma}
\newtheorem{prop}[thm]{Proposition}
\newtheorem{cor}[thm]{Corollary}
\theoremstyle{definition}
\newtheorem{exam}[thm]{Example}
\newtheorem{defn}[thm]{Definition}
\newtheorem{conj}[thm]{Conjecture}
\newtheorem{question}[thm]{Question}
\newtheorem{problem}[thm]{Problem}
\newtheorem{remark}[thm]{Remark}
\newtheorem*{note}{Note}

% DO NOT DELETE THIS COMMENT!!! MACROS BELOW:
\newcommand\Gal{\operatorname{Gal}}
\newcommand\NI{\operatorname{NI}}
\newcommand\LRmin{\operatorname{LRmin}}
\newcommand\pts[1]{\textbf{[#1~points]}}
\newcommand\Pts[1]{\quad\textbf{[#1~points]}}
\newcommand\evencycle{\operatorname{evencycle}}
\newcommand\inv{\operatorname{inv}}
\newcommand\cycle{\operatorname{cycle}}
\newcommand\Motz{\operatorname{Motz}}
\newcommand\Fix{\operatorname{Fix}}
\newcommand\sgn{\operatorname{sgn}}
\newcommand\sym{\mathfrak{S}}
\newcommand\invol{\mathfrak{I}}
\newcommand\NN{\mathbb{N}}
\newcommand\QQ{\mathbb{Q}}
\newcommand{\CC}{\mathbb{C}}
\newcommand{\ZZ}{\mathbb{Z}}
\newcommand{\RR}{\mathbb{R}}
\newcommand\LL{\mathcal{L}}
\newcommand\FF{\mathbb{F}}
\newcommand\LH{\operatorname{LH}}
\newcommand\CH{\operatorname{CH}}

\newcommand\Mot{\operatorname{Mot}}
\newcommand{\Dyck}{\operatorname{Dyck}}

\newcommand\Par{\operatorname{Par}}
\newcommand\RPP{\operatorname{RPP}}
\newcommand\SSYT{\operatorname{SSYT}}
\newcommand\SYT{\operatorname{SYT}}

\newcommand\wt{\operatorname{wt}}

\renewcommand\vec[1]{\bm{#1}}
\newcommand\vx{\vec{x}}
\newcommand\vb{\vec{b}}
\newcommand\vla{\vec{\lambda}}
\newcommand\flr[1]{\left\lfloor #1\right\rfloor}
\newcommand\Qbinom[3]{\genfrac{[}{]}{0pt}{}{#1}{#2}_{#3}}
\newcommand\qbinom[2]{\Qbinom{#1}{#2}{q}}

\newcommand\hyper[5]{{}_{#1}F_{#2} \left(#3;#4;#5\right)}
\newcommand\qhyper[5]{{}_{#1}\phi_{#2} \left(#3;#4;#5\right)}
\newcommand\Hyper[5]{{}_{#1}F_{#2} \left( \left.
    \begin{matrix}
      #3\\
      #4\\
    \end{matrix}
    \:\right|\: #5
    \right)}
\newcommand\qHyper[5]{{}_{#1}\phi_{#2} \left(
    \begin{matrix}
      #3\\
      #4\\
    \end{matrix}
    ; #5
    \right)}

\newcommand\comment[1]{\textcolor{blue}{\bf #1}}

%%%%%%%%%%%%%%%%%%%%%%%%%%%%%%%%%%%%%%%%%%%%%%%%%%%%%%%%%%%%%%%%

\title{Homework}


\begin{document}

\maketitle
\tableofcontents

\newpage
\section{Homework 1 (Due: Apr 5)}

\begin{problem}
 Let \(A\) be a nonempty set and let \(k\) be a positive integer with \(k \leq|A|\). The symmetric group \(S_A\) acts on the set \(B\) consisting of all subsets of \(A\) of cardinality \(k\) by 
 \[
   \sigma \cdot\left\{a_1, \ldots, a_k\right\}= \left\{\sigma\left(a_1\right), \ldots, \sigma\left(a_k\right)\right\}.
 \]
 \begin{enumerate}
 \item Prove that this is a group action.
 \item Describe explicitly how the elements (1 2) and (1 2 3) act on the six 2-element subsets of \(\{1,2,3,4\}\).
 \end{enumerate}
\end{problem}

% \begin{proof}[Solution]\
% \begin{center}
%   \includegraphics[scale=.2]{./figures/image1.png}
% \end{center}
% \end{proof}


\begin{problem}
  Let \(H\) be a group acting on a set \(A\). Prove that the relation
  \(\sim\) on \(A\) defined by \(a \sim b \) if and only if
  \( a=h b \) for some \(h \in H\) is an equivalence
  relation. (For each \(x \in A\) the equivalence class of \(x\) under
  \(\sim\) is called the orbit of \(x\) under the action of \(H\). The
  orbits under the action of \(H\) partition the set \(A\).)
\end{problem}
% \begin{proof}[Solution]\
% \begin{center}
%   \includegraphics[scale=.25]{./figures/image2.png}
% \end{center}
% \end{proof}



\begin{problem}
 In each of parts (1) to (5) give the number of nonisomorphic abelian groups of the specified order - do not list the groups:
\begin{enumerate}
\item order 100
\item order 576
\item order 1155
\item order 42875
\item order 2704
\end{enumerate}
\end{problem}

% \begin{proof}[Solution]\
% \begin{center}
%   \includegraphics[scale=.15]{./figures/image3.png}
% \end{center}
% \end{proof}


\begin{problem}
  In each of parts (1) to (5) give the lists of invariant factors for all abelian groups of the specified order:
\begin{enumerate}
\item order 270
\item order 9801
\item order 320
\item order 105
\item order 44100
\end{enumerate}
\end{problem}

% \begin{proof}[Solution]\
% \begin{center}
%   \includegraphics[scale=.2]{./figures/image4.png}
% \end{center}
% \end{proof}


\begin{problem}
  In each of parts (1) to (5) give the lists of elementary divisors
  for all abelian groups of the specified order and then match each
  list with the corresponding list of invariant factors found
  in the preceding problem:
\begin{enumerate}
\item order 270
\item order 9801
\item order 320
\item order 105
\item order 44100
\end{enumerate}
\end{problem}

% \begin{proof}[Solution]\
% \begin{center}
%   \includegraphics[scale=.2]{./figures/image5.png}
% \end{center}
% \end{proof}



\begin{problem}
  Let \(R\) be a ring with identity and let \(S\) be a subring of
  \(R\) containing the identity. Prove that if \(u\) is a unit in
  \(S\) then \(u\) is a unit in \(R\). Show by example that the
  converse is false.
\end{problem}

% \begin{proof}[Solution]\
% \begin{center}
%   \includegraphics[scale=.2]{./figures/image6.png}
% \end{center}
% \end{proof}


\begin{problem}
  Let \( R \) be a ring with \( 1\ne 0 \).
  \begin{enumerate}
  \item Prove that if \( a \) is a zero divisor, then it is not a
    unit.
  \item Prove that if \( ab=ac \) and \( a\ne0 \) is not a zero divisor,
  then \( b=c \).
  \end{enumerate}
\end{problem}

% \begin{proof}[Solution]\
% \begin{center}
%   \includegraphics[scale=.2]{./figures/image7.png}
% \end{center}
% \end{proof}


\begin{problem}
 Assume \(R\) is commutative with \( 1\ne0 \). Prove that if \(P\) is a prime ideal of \(R\) and \(P\) contains no zero divisors then \(R\) is an integral domain. 
\end{problem}

% \begin{proof}[Solution]
% \begin{center}
%   \includegraphics[scale=.2]{./figures/image8.png}
% \end{center}
% \end{proof}


\begin{problem}
  Let \( R \) be a ring with \( 1\ne 0 \).
  Let \(A=\left(a_1, a_2, \ldots, a_n\right)\) be a nonzero finitely
  generated ideal of \(R\). Prove that there is an ideal \(B\) which
  is maximal with respect to the property that it does not contain
  \(A\). [Use Zorn’s Lemma.]
\end{problem}

% \begin{proof}[Solution]\
% \begin{center}
%   \includegraphics[scale=.2]{./figures/image9.png}
% \end{center}
% \end{proof}


\begin{problem}
Let \(n_1, n_2, \ldots, n_k\) be integers which are relatively prime in pairs: \(\left(n_i, n_j\right)=1\) for all \(i \neq j\).
\begin{enumerate}
\item Show that the Chinese Remainder Theorem implies that for any \(a_1, \ldots, a_k \in \mathbb{Z}\) there is a solution \(x \in \mathbb{Z}\) to the simultaneous congruences
\[
x \equiv a_1 \bmod n_1, \quad x \equiv a_2 \bmod n_2, \quad \ldots, \quad x \equiv a_k \bmod n_k
\]
and that the solution \(x\) is unique \(\bmod n = n_1 n_2 \ldots n_k\).
\item Let \(n_i^{\prime}=n / n_i\) be the quotient of \(n\) by \(n_i\), which is relatively prime to \(n_i\) by assumption. Let \(t_i\) be the inverse of \(n_i^{\prime} \bmod n_i\). Prove that the solution \(x\) in (a) is given by
\[
x=a_1 t_1 n_1^{\prime}+a_2 t_2 n_2^{\prime}+\cdots+a_k t_k n_k^{\prime} \bmod n .
\]
Note that the elements \(t_i\) can be quickly found by the Euclidean Algorithm as described in Section 2 of the Preliminaries chapter (writing \(a n_i+b n_i^{\prime}=\left(n_i, n_i^{\prime}\right)=1\) gives \(t_i=b\) ) and that these then quickly give the solutions to the system of congruences above for any choice of \(a_1, a_2, \ldots, a_k\).

\item Solve the simultaneous system of congruences
\[
x \equiv 1 \bmod 8, \quad x \equiv 2 \bmod 25, \quad \text { and } x \equiv 3 \bmod 81
\]
and the simultaneous system
\[
y \equiv 5 \bmod 8, \quad y \equiv 12 \bmod 25, \quad \text { and } \quad y \equiv 47 \bmod 81
\]
\end{enumerate}
\end{problem}

% \begin{proof}[Solution]\
% \newpage
% \begin{center}
%   \includegraphics[scale=.2]{./figures/image10.png}
% \end{center}
% \end{proof}



\newpage
\section{Homework 2 (Due: Apr 19)}

For all problems, suppose that \( R \) is a ring with \( 1\ne0 \)
and \( M \) is a left \( R \)-module.

\begin{problem}
  An element \(m\) of the \(R\)-module \(M\) is called a \emph{torsion
  element} if \(r m=0\) for some nonzero element \(r \in R\). The set
  of torsion elements is denoted
\[
\operatorname{Tor}(M)=\{m \in M \mid r m=0 \text { for some nonzero } r \in R\} .
\]
\begin{enumerate}
\item Prove that if \(R\) is an integral domain then \(\operatorname{Tor}(M)\) is a submodule of \(M\) (called the torsion submodule of \(M\) ).
\item  Give an example of a ring \(R\) and an \(R\)-module \(M\) such that \(\operatorname{Tor}(M)\) is not a submodule. [Consider the torsion elements in the \(R\)-module \(R\).]
\item If \(R\) has zero divisors show that every nonzero \(R\)-module has nonzero torsion elements.
\end{enumerate}
\end{problem}

% \begin{proof}[Solution]\
% \begin{center}
%   \includegraphics[scale=.2]{./figures/image12.png}
% \end{center}
% \end{proof}


\begin{problem}
  \begin{enumerate}
  \item If \(N\) is a submodule of \(M\), the \emph{annihilator of \(N\) in \(R\) } is defined to be \(\{r \in R \mid r n=0\) for all
    \(n \in N\}\). Prove that the annihilator of \(N\) in \(R\) is a
    2-sided ideal of \(R\).
  \item If \(I\) is a right ideal of \(R\), the \emph{annihilator of \(I\) in \(M\)} is defined to be \{ \(m \in M \mid a m=0\) for all \(a \in I\}\). Prove that the annihilator of \(I\) in \(M\) is a submodule of \(M\).
  \item Let \(M\) be the abelian group (i.e., \(\mathbb{Z}\)-module) \(\mathbb{Z} / 24 \mathbb{Z} \times \mathbb{Z} / 15 \mathbb{Z} \times \mathbb{Z} / 50 \mathbb{Z}\).
    \begin{enumerate}
    \item Find the annihilator of \(M\) in \(\mathbb{Z}\) (i.e., a generator for this principal ideal).
    \item Let \(I=2 \mathbb{Z}\). Describe the annihilator of \(I\) in \(M\) as a direct product of cyclic groups.
    \end{enumerate}
  \end{enumerate}
\end{problem}

% \begin{proof}[Solution]\
% \begin{center}
%   \includegraphics[scale=.19]{./figures/image13.png}
% \end{center}
% \end{proof}


\begin{problem}
  \begin{enumerate}
  \item Let \(F=\mathbb{R}\), let \(V=\mathbb{R}^2\) and let \(T\) be the linear transformation from \(V\) to \(V\) which is rotation clockwise about the origin by \(\pi / 2\) radians. Show that \(V\) and 0 are the only \(F[x]\)-submodules for this \(T\).
  \item Let \(F=\mathbb{R}\), let \(V=\mathbb{R}^2\) and let \(T\) be the linear transformation from \(V\) to \(V\) which is projection onto the \(y\)-axis. Show that \(V, 0\), the \(x\)-axis and the \(y\)-axis are the only \(F[x]-\) submodules for this \(T\).
  \item Let \(F=\mathbb{R}\), let \(V=\mathbb{R}^2\) and let \(T\) be the linear transformation from \(V\) to \(V\) which is rotation clockwise about the origin by \(\pi\) radians. Show that every subspace of \(V\) is an
\(F[x]\)-submodule for this \(T\).
  \end{enumerate}
\end{problem}

% \begin{proof}[Solution]\
% \begin{center}
%   \includegraphics[scale=.2]{./figures/image14.png}
% \end{center}
% \end{proof}


\begin{problem}
  \begin{enumerate}
    \item For any left ideal \(I\) of \(R\) define
\[
I M=\left\{\sum_{\text {finite }} a_i m_i \mid a_i \in I, m_i \in M\right\}
\]
to be the collection of all finite sums of elements of the form \(a m\) where \(a \in I\) and \(m \in M\). Prove that \(I M\) is a submodule of \(M\).
  \item Let \(A_1, A_2, \ldots, A_n\) be \(R\)-modules and let \(B_i\) be a submodule of \(A_i\) for each \(i=1,2, \ldots, n\). Prove that
\[
\left(A_1 \times \cdots \times A_n\right) /\left(B_1 \times \cdots \times B_n\right) \cong\left(A_1 / B_1\right) \times \cdots \times\left(A_n / B_n\right) .
\]
\item Let \(I\) be a left ideal of \(R\) and let \(n\) be a positive integer. Prove that
\[
R^n / I R^n \cong R / I R \times \cdots \times R / I R \quad(n \text { times) }.
\]
\item   Assume \(R\) is commutative. Prove that \(R^n \cong R^m\) if and
  only if \(n=m\), i.e., two free \(R\)-modules of finite rank are
  isomorphic if and only if they have the same rank. [Apply the
  previous problem with \(I\) a maximal ideal of \(R\). You may use
  the fact that if \(F\) is a field, then \(F^n \cong F^m\) if and
  only if \(n=m\).]
  \end{enumerate}
\end{problem}

% \begin{proof}[Solution]\
% \begin{center}
%   \includegraphics[scale=.2]{./figures/image15.png}
% \end{center}
% \end{proof}



\begin{problem}
  Let \(I\) be a nonempty index set and for each \(i \in I\) let \(M_i\) be an \(R\)-module. The direct product of the modules \(M_i\) is defined to be their direct product as abelian groups (cf. Exercise 15 in Section 5.1) with the action of \(R\) componentwise multiplication. The direct sum of the modules \(M_i\) is defined to be the restricted direct product of the abelian groups \(M_i\) (cf. Exercise 17 in Section 5.1) with the action of \(R\) componentwise multiplication. In other words, the direct sum of the \(M_i\)'s is the subset of the direct product, \(\prod_{i \in I} M_i\), which consists of all elements \(\prod_{i \in I} m_i\) such that only finitely many of the components \(m_i\) are nonzero; the action of \(R\) on the direct product or direct sum is given by \(r \prod_{i \in I} m_i=\prod_{i \in I} r m_i\) (cf. Appendix I for the definition of Cartesian products of infinitely many sets). The direct sum will be denoted by \(\oplus_{i \in I} M_i\).
  \begin{enumerate}
  \item Prove that the direct product of the \(M_i\)'s is an \(R\)-module and the direct sum of the \(M_i\)'s is a submodule of their direct product.
  \item Show that if \(R=\mathbb{Z}, I=\mathbb{Z}^{+}\)and \(M_i\) is the cyclic group of order \(i\) for each \(i\), then the direct sum of the \(M_i\) 's is not isomorphic to their direct product. [Look at torsion.]
  \end{enumerate}
\end{problem}

% \begin{proof}[Solution]\
% \begin{center}
%   \includegraphics[scale=.2]{./figures/image16.png}
% \end{center}
% \end{proof}


% \begin{problem}
%   (An arbitrary direct product of free modules need not be free) For each positive integer \(i\) let \(M_i\) be the free \(\mathbb{Z}\)-module \(\mathbb{Z}\), and let \(M\) be the direct product \(\prod_{i \in \mathbb{Z}^{+}} M_i\). Each element of \(M\) can be written uniquely in the form \(\left(a_1, a_2, a_3, \ldots\right)\) with \(a_i \in \mathbb{Z}\) for all \(i\). Let \(N\) be the submodule of \(M\) consisting of all such tuples with only finitely many nonzero \(a_i\). Assume \(M\) is a free \(\mathbb{Z}\)-module with basis \(\mathcal{B}\).
%   \begin{enumerate}
%   \item Show that \(N\) is countable.
% \item Show that there is some countable subset \(\mathcal{B}_1\) of \(\mathcal{B}\) such that \(N\) is contained in the submodule, \(N_1\), generated by \(\mathcal{B}_1\). Show also that \(N_1\) is countable.
% \item Let \(\bar{M}=M / N_1\). Show that \(\bar{M}\) is a free \(\mathbb{Z}\)-module. Deduce that if \(\bar{x}\) is any nonzero element of \(\bar{M}\) then there are only finitely many distinct positive integers \(k\) such that \(\bar{x}=k \bar{m}\) for some \(m \in M\) (depending on \(k\) ).
% \item Let \(\mathcal{S}=\left\{\left(b_1, b_2, b_3, \ldots\right) \mid b_i= \pm i!\right.\) for all \(\left.i\right\}\). Prove that \(\mathcal{S}\) is uncountable. Deduce that there is some \(s \in \mathcal{S}\) with \(s \notin N_1\).
% \item Show that the assumption \(M\) is free leads to a contradiction: By (4) we may choose \(s \in \mathcal{S}\) with \(s \notin N_1\). Show that for each positive integer \(k\) there is some \(m \in M\) with \(\bar{s}=k \bar{m}\), contrary to (3). [Use the fact that \(N \subseteq N_1\).]
%   \end{enumerate}

% \end{problem}

% \begin{problem}
%   (Free modules over noncommutative rings need not have a unique rank) Let \(M\) be the \(\mathbb{Z}\)-module \(\mathbb{Z} \times \mathbb{Z} \times \cdots\) of Exercise 24 and let \(R\) be its endomorphism \(\operatorname{ring}, R=\operatorname{End}_{\mathbb{Z}}(M)\) (cf. Exercises 29 and 30 in Section 7.1). Define \(\varphi_1, \varphi_2 \in R\) by
% \[
% \begin{aligned}
% & \varphi_1\left(a_1, a_2, a_3, \ldots\right)=\left(a_1, a_3, a_5, \ldots\right) \\
% & \varphi_2\left(a_1, a_2, a_3, \ldots\right)=\left(a_2, a_4, a_6, \ldots\right)
% \end{aligned}
% \]
% \begin{enumerate}
% \item Prove that \(\left\{\varphi_1, \varphi_2\right\}\) is a free basis of the left \(R\)-module \(R\). [Define the maps \(\psi_1\) and \(\psi_2\) by \(\psi_1\left(a_1, a_2, \ldots\right)=\left(a_1, 0, a_2, 0, \ldots\right)\) and \(\psi_2\left(a_1, a_2, \ldots\right)=\left(0, a_1, 0, a_2, \ldots\right)\). Verify that \(\varphi_i \psi_i=1, \varphi_1 \psi_2=0=\varphi_2 \psi_1\) and \(\psi_1 \varphi_1+\psi_2 \varphi_2=1\). Use these relations to prove that \(\varphi_1, \varphi_2\) are independent and generate \(R\) as a left \(R\)-module.]
% \item Use (1) to prove that \(R \cong R^2\) and deduce that \(R \cong R^n\) for all \(n \in \mathbb{Z}^{+}\).
% \end{enumerate}
% \end{problem}


\begin{problem}
  \begin{enumerate}
  \item Show that the element ``\(2 \otimes 1\)'' is 0 in \(\mathbb{Z} \otimes_{\mathbb{Z}} \mathbb{Z} / 2 \mathbb{Z}\) but is nonzero in \(2 \mathbb{Z} \otimes_{\mathbb{Z}} \mathbb{Z} / 2 \mathbb{Z}\).
  \item Show that \(\mathbb{C} \otimes_{\mathbb{R}} \mathbb{C}\) and \(\mathbb{C} \otimes_{\mathbb{C}} \mathbb{C}\) are both left \(\mathbb{R}\)-modules but are not isomorphic as \(\mathbb{R}\)-modules.
  \item Show that \(\mathbb{Q} \otimes_{\mathbb{Z}} \mathbb{Q}\) and \(\mathbb{Q} \otimes_{\mathbb{Q}} \mathbb{Q}\) are isomorphic left \(\mathbb{Q}\)-modules. [Show they are both 1-dimensional vector spaces over \(\mathbb{Q}\).]
  \item   If \(R\) is any integral domain with quotient field \(Q\), prove that \((Q / R) \otimes_R(Q / R)=0\).
  \item   Let \(\left\{e_1, e_2\right\}\) be a basis of \(V=\mathbb{R}^2\). Show that the element \(e_1 \otimes e_2+e_2 \otimes e_1\) in \(V \otimes_{\mathbb{R}} V\) cannot be written as a simple tensor \(v \otimes w\) for any \(v, w \in \mathbb{R}^2\).
  \end{enumerate}
\end{problem}

% \begin{proof}[Solution]\
% \begin{center}
%   \includegraphics[scale=.2]{./figures/image17.png}
% \end{center}
% \end{proof}



\begin{problem}
  Suppose \(R\) is commutative and \(N \cong R^n\) is a free \(R\)-module of rank \(n\) with \(R\)-module basis \(e_1, \ldots, e_n\).
  \begin{enumerate}
  \item For any nonzero \(R\)-module \(M\) show that every element of \(M \otimes N\) can be written uniquely in the form \(\sum_{i=1}^n m_i \otimes e_i\) where \(m_i \in M\). Deduce that if \(\sum_{i=1}^n m_i \otimes e_i=0\) in \(M \otimes N\) then \(m_i=0\) for \(i=1, \ldots, n\).
\item Show that if \(\sum m_i \otimes n_i=0\) in \(M \otimes N\) where the \(n_i\) are merely assumed to be \(R\) linearly independent then it is not necessarily true that all the \(m_i\) are 0 . [Consider \(R=\mathbb{Z}, n=1, M=\mathbb{Z} / 2 \mathbb{Z}\), and the element \(1 \otimes 2\).
  \end{enumerate}
\end{problem}

% \begin{proof}[Solution]\
% \begin{center}
%   \includegraphics[scale=.2]{./figures/image18.png}
% \end{center}
% \end{proof}


% \begin{problem}
%   Let \(I=(2, x)\) be the ideal generated by 2 and \(x\) in the ring \(R=\mathbb{Z}[x]\). The ring \(\mathbb{Z} / 2 \mathbb{Z}=R / I\) is naturally an \(R\)-module annihilated by both 2 and \(x\).
% \begin{enumerate}
% \item Show that the map \(\varphi: I \times I \rightarrow \mathbb{Z} / 2 \mathbb{Z}\) defined by
% \[
% \varphi\left(a_0+a_1 x+\cdots+a_n x^n, b_0+b_1 x+\cdots+b_m x^m\right)=\frac{a_0}{2} b_1 \bmod 2
% \]
% is \(R\)-bilinear.
% \item Show that there is an \(R\)-module homomorphism from \(I \otimes_R I \rightarrow \mathbb{Z} / 2 \mathbb{Z}\) mapping \(p(x) \otimes q(x)\) to \(\frac{p(0)}{2} q^{\prime}(0)\) where \(q^{\prime}\) denotes the usual polynomial derivative of \(q\).
% \item Show that \(2 \otimes x \neq x \otimes 2\) in \(I \otimes_R I\).
% \end{enumerate}
% \end{problem}



\begin{problem}
 Suppose that
 \begin{center}
  \includegraphics[scale=.15]{./figures/image11.png}
\end{center}
is a commutative diagram of groups and that the rows are exact. Prove that
\begin{enumerate}
\item if \(\varphi\) and \(\alpha\) are surjective, and \(\beta\) is injective then \(\gamma\) is injective. [If \(c \in \operatorname{ker} \gamma\), show there is a \(b \in B\) with \(\varphi(b)=c\). Show that \(\varphi^{\prime}(\beta(b))=0\) and deduce that \(\beta(b)=\psi^{\prime}\left(a^{\prime}\right)\) for some \(a^{\prime} \in A^{\prime}\). Show there is an \(a \in A\) with \(\alpha(a)=a^{\prime}\) and that \(\beta(\psi(a))=\beta(b)\). Conclude that \(b=\psi(a)\) and hence \(c=\varphi(b)=0\).]
\item if \(\psi^{\prime}, \alpha\), and \(\gamma\) are injective, then \(\beta\) is injective,
\item if \(\varphi, \alpha\), and \(\gamma\) are surjective, then \(\beta\) is surjective,
\item if \(\beta\) is injective, \(\alpha\) and \(\varphi\) are surjective, then \(\gamma\) is injective,
\item if \(\beta\) is surjective, \(\gamma\) and \(\psi^{\prime}\) are injective, then \(\alpha\) is surjective.
\end{enumerate}
\end{problem}

% \begin{proof}[Solution]\
% \begin{center}
%   \includegraphics[scale=.2]{./figures/image19.png}
% \end{center}
% \end{proof}





\begin{problem}
   Let \(P_1\) and \(P_2\) be \(R\)-modules. Prove that \(P_1 \oplus P_2\) is a projective \(R\)-module if and only if both \(P_1\) and \(P_2\) are projective.
  % \begin{enumerate}
  % \item Let \(Q_1\) and \(Q_2\) be \(R\)-modules. Prove that \(Q_1 \oplus Q_2\) is an injective \(R\)-module if and only if both \(Q_1\) and \(Q_2\) are injective.
  % \item Let \(A_1\) and \(A_2\) be \(R\)-modules. Prove that \(A_1 \oplus A_2\) is a flat \(R\)-module if and only if both \(A_1\) and \(A_2\) are flat. More generally, prove that an arbitrary direct sum \(\sum A_i\) of \(R\)-modules is flat if and only if each \(A_i\) is flat. [Use the fact that tensor product commutes with arbitrary direct sums.]
  % \end{enumerate}
\end{problem}

% \begin{proof}[Solution]\
% \begin{center}
%   \includegraphics[scale=.2]{./figures/image20.png}
% \end{center}
% \end{proof}



\begin{problem}
 Let \(0 \longrightarrow L \xrightarrow{\psi} M \xrightarrow{\varphi} N \longrightarrow 0\) be a sequence of \(R\)-modules.
\begin{enumerate}
 \item Prove that the associated sequence
\[
0 \longrightarrow \operatorname{Hom}_R(D, L) \xrightarrow{\psi^{\prime}} \operatorname{Hom}_R(D, M) \xrightarrow{\varphi^{\prime}} \operatorname{Hom}_R(D, N) \longrightarrow 0
\]
is a short exact sequence of abelian groups for all \(R\)-modules \(D\) if and only if the original sequence is a split short exact sequence. [To show the sequence splits, take \(D=N\) and show the lift of the identity map in \(\operatorname{Hom}_R(N, N)\) to \(\operatorname{Hom}_R(N, {M})\) is a splitting homomorphism for \(\varphi\).]
\item Prove that the associated sequence
\[
0 \longrightarrow \operatorname{Hom}_R(N, D) \xrightarrow{\varphi^{\prime}} \operatorname{Hom}_R(M, D) \xrightarrow{\psi^{\prime}} \operatorname{Hom}_R(L, D) \longrightarrow 0
\]
is a short exact sequence of abelian groups for all \(R\)-modules \(D\) if and only if the original sequence is a split short exact sequence.
\end{enumerate}
\end{problem}

% \begin{proof}[Solution]\
%   (1)
%   \begin{center}
%   \includegraphics[scale=.2]{./figures/image21.png}
% \end{center}

% (2)
% \begin{center}
%   \includegraphics[scale=.2]{./figures/image22.png}
%   \includegraphics[scale=.2]{./figures/image23.png}
% \end{center}
% \end{proof}


\newpage
\section{Homework 3 (Due: May 10)}

\begin{problem}[Section 12.1, Exercise 2]
 Let \(M\) be a module over the integral domain \(R\).
 \begin{enumerate}
 \item Suppose that \(M\) has rank \(n\) and that \(x_1, x_2, \ldots, x_n\) is any maximal set of linearly independent elements of \(M\). Let \(N=R x_1+\cdots+R x_n\) be the submodule generated by \(x_1, x_2, \ldots, x_n\). Prove that \(N\) is isomorphic to \(R^n\) and that the quotient \(M / N\) is a torsion \(R\)-module (equivalently, the elements \(x_1, \ldots, x_n\) are linearly independent and for any \(y \in M\) there is a nonzero element \(r \in R\) such that \(r y\) can be written as a linear combination \(r_1 x_1+\cdots+r_n x_n\) of the \(x_i\) ).
\item Prove conversely that if \(M\) contains a submodule \(N\) that is free of rank \(n\) (i.e., \(N \cong\) \(R^n\) ) such that the quotient \(M / N\) is a torsion \(R\)-module then \(M\) has rank \(n\). [Let \(y_1, y_2, \ldots, y_{n+1}\) be any \(n+1\) elements of \(M\). Use the fact that \(M / N\) is torsion to write \(r_i y_i\) as a linear combination of a basis for \(N\) for some nonzero elements \(r_1, \ldots, r_{n+1}\) of \(R\). Use an argument as in the proof of Proposition 3 to see that the \(r_i y_i\), and hence also the \(y_i\), are linearly dependent.] 
 \end{enumerate}
\end{problem}

% \begin{proof}[Solution]\
%   \begin{center}
%   \includegraphics[scale=.3]{./figures/image24.png}
% \end{center}
% \end{proof}


\begin{problem}[Section 12.1, Exercise 3]
 Let \(R\) be an integral domain and let \(A\) and \(B\) be \(R\)-modules of ranks \(m\) and \(n\), respectively. Prove that the rank of \(A \oplus B\) is \(m+n\). [Use the previous exercise.] 
\end{problem}
% \begin{proof}[Solution]\
% \begin{center}
%   \includegraphics[scale=.2]{./figures/image25.png}
% \end{center}
% \end{proof}



\begin{problem}[Section 12.1, Exercise 4]
Let \(R\) be an integral domain, let \(M\) be an \(R\)-module and let \(N\) be a submodule of \(M\). Suppose \(M\) has rank \(n, N\) has rank \(r\) and the quotient \(M / N\) has rank \(s\). Prove that \(n=r+s\). [Let \(x_1, x_2, \ldots, x_s\) be elements of \(M\) whose images in \(M / N\) are a maximal set of independent elements and let \(x_{s+1}, x_{s+2}, \ldots, x_{s+r}\) be a maximal set of independent elements in \(N\). Prove that \(x_1, x_2, \ldots, x_{s+r}\) are linearly independent in \(M\) and that for any element \(y \in M\) there is a nonzero element \(r \in R\) such that \(r y\) is a linear combination of these elements. Then use Exercise 2.]  
\end{problem}

% \begin{proof}[Solution]\
% \begin{center}
%   \includegraphics[scale=.2]{./figures/image26.png}
% \end{center}
% \end{proof}



\begin{problem}[Section 12.1, Exercise 5]
 Let \(R=\mathbb{Z}[x]\) and let \(M=(2, x)\) be the ideal generated by 2 and \(x\), considered as a submodule of \(R\). Show that \(\{2, x\}\) is not a basis of \(M\). [Find a nontrivial \(R\)-linear dependence between these two elements.] Show that the rank of \(M\) is 1 but that \(M\) is not free of rank 1 (cf. Exercise 2). 
\end{problem}

% \begin{proof}[Solution]\
% \begin{center}
%   \includegraphics[scale=.2]{./figures/image27.png}
% \end{center}
% \end{proof}



\begin{problem}[Section 12.1, Exercise 6]
 Show that if \(R\) is an integral domain and \(M\) is any nonprincipal ideal of \(R\) then \(M\) is torsion free of rank 1 but is not a free \(R\)-module. 
\end{problem}

% \begin{proof}[Solution]\
% \begin{center}
%   \includegraphics[scale=.2]{./figures/image28.png}
% \end{center}
% \end{proof}



\begin{problem}[Section 12.1, Exercise 7]
 Let \(R\) be any ring, let \(A_1, A_2, \ldots, A_m\) be \(R\)-modules and let \(B_i\) be a submodule of \(A_i\), \(1 \leq i \leq m\). Prove that
\[
\left(A_1 \oplus A_2 \oplus \cdots \oplus A_m\right) /\left(B_1 \oplus B_2 \oplus \cdots \oplus B_m\right) \cong\left(A_1 / B_1\right) \oplus\left(A_2 / B_2\right) \oplus \cdots \oplus\left(A_m / B_m\right)
\] 
\end{problem}

% \begin{proof}[Solution]\
% \begin{center}
%   \includegraphics[scale=.2]{./figures/image29.png}
% \end{center}
% \end{proof}


\begin{problem}[Section 12.1, Exercise 8]
 Let \(R\) be a P.I.D., let \(B\) be a torsion \(R\)-module and let \(p\) be a prime in \(R\). Prove that if \(p b=0\) for some nonzero \(b \in B\), then \(\operatorname{Ann}(B) \subseteq(p)\). 
\end{problem}

% \begin{proof}[Solution]\
% \begin{center}
%   \includegraphics[scale=.2]{./figures/image30.png}
% \end{center}
% \end{proof}



\begin{problem}[Section 12.1, Exercise 9]
 Give an example of an integral domain \(R\) and a nonzero torsion \(R\)-module \(M\) such that \(\operatorname{Ann}(M)=0\). Prove that if \(N\) is a finitely generated torsion \(R\)-module then \(\operatorname{Ann}(N) \neq 0\). 
\end{problem}

% \begin{proof}[Solution]\
% \begin{center}
%   \includegraphics[scale=.2]{./figures/image31.png}
% \end{center}
% \end{proof}



\begin{problem}[Section 12.1, Exercise 10]
 For \(p\) a prime in the P.I.D. \(R\) and \(N\) a torsion \(R\)-module prove that the \(p\)-primary component of \(N\) is a submodule of \(N\) and prove that \(N\) is the direct sum of its \(p\)-primary components (there need not be finitely many of them). 
\end{problem}

% \begin{proof}[Solution]\
% \begin{center}
%   \includegraphics[scale=.2]{./figures/image32.png}
% \end{center}
% \end{proof}


\begin{problem}[Section 12.1, Exercise 15]
Prove that if \(R\) is a Noetherian ring then \(R^n\) is a Noetherian \(R\)-module. [Fix a basis of \(R^n\). If \(M\) is a submodule of \(R^n\) show that the collection of first coordinates of elements of \(M\) is a submodule of \(R\) hence is finitely generated. Let \(m_1, m_2, \ldots, m_k\) be elements of \(M\)
whose first coordinates generate this submodule of \(R\). Show that any element of \(M\) can be written as an \(R\)-linear combination of \(m_1, m_2, \ldots, m_k\) plus an element of \(M\) whose first coordinate is 0 . Prove that \(M \cap R^{n-1}\) is a submodule of \(R^{n-1}\) where \(R^{n-1}\) is the set of elements of \(R^n\) with first coordinate 0 and then use induction on \(n\).
\end{problem}

% \begin{proof}[Solution]\
% \begin{center}
%   \includegraphics[scale=.2]{./figures/image33.png}
% \end{center}
% \end{proof}


\end{document}
