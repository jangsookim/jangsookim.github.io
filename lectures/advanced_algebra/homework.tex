\documentclass{amsart}

\usepackage{amsmath,amsthm,amssymb,bm}
\usepackage{hyperref}
\usepackage{a4wide}
\usepackage{cleveref}
% \usepackage{refcheck}
\usepackage{graphicx,color}
\usepackage{tikz}
\numberwithin{equation}{section}

\newtheorem{thm}{Theorem}[section]
\newtheorem{lem}[thm]{Lemma}
\newtheorem{prop}[thm]{Proposition}
\newtheorem{cor}[thm]{Corollary}
\theoremstyle{definition}
\newtheorem{exam}[thm]{Example}
\newtheorem{defn}[thm]{Definition}
\newtheorem{conj}[thm]{Conjecture}
\newtheorem{question}[thm]{Question}
\newtheorem{problem}[thm]{Problem}
\newtheorem{remark}[thm]{Remark}
\newtheorem*{note}{Note}

% DO NOT DELETE THIS COMMENT!!! MACROS BELOW:
\newcommand\NI{\operatorname{NI}}
\newcommand\LRmin{\operatorname{LRmin}}
\newcommand\pts[1]{\textbf{[#1~points]}}
\newcommand\Pts[1]{\quad\textbf{[#1~points]}}
\newcommand\evencycle{\operatorname{evencycle}}
\newcommand\inv{\operatorname{inv}}
\newcommand\cycle{\operatorname{cycle}}
\newcommand\Motz{\operatorname{Motz}}
\newcommand\Fix{\operatorname{Fix}}
\newcommand\sgn{\operatorname{sgn}}
\newcommand\sym{\mathfrak{S}}
\newcommand\invol{\mathfrak{I}}
\newcommand\NN{\mathbb{N}}
\newcommand\QQ{\mathbb{Q}}
\newcommand{\CC}{\mathbb{C}}
\newcommand{\ZZ}{\mathbb{Z}}
\newcommand{\RR}{\mathbb{R}}
\newcommand\LL{\mathcal{L}}
\newcommand\FF{\mathbb{F}}
\newcommand\LH{\operatorname{LH}}
\newcommand\CH{\operatorname{CH}}

\newcommand\Mot{\operatorname{Mot}}
\newcommand{\Dyck}{\operatorname{Dyck}}

\newcommand\Par{\operatorname{Par}}
\newcommand\RPP{\operatorname{RPP}}
\newcommand\SSYT{\operatorname{SSYT}}
\newcommand\SYT{\operatorname{SYT}}

\newcommand\wt{\operatorname{wt}}

\renewcommand\vec[1]{\bm{#1}}
\newcommand\vx{\vec{x}}
\newcommand\vb{\vec{b}}
\newcommand\vla{\vec{\lambda}}
\newcommand\flr[1]{\left\lfloor #1\right\rfloor}
\newcommand\Qbinom[3]{\genfrac{[}{]}{0pt}{}{#1}{#2}_{#3}}
\newcommand\qbinom[2]{\Qbinom{#1}{#2}{q}}

\newcommand\hyper[5]{{}_{#1}F_{#2} \left(#3;#4;#5\right)}
\newcommand\qhyper[5]{{}_{#1}\phi_{#2} \left(#3;#4;#5\right)}
\newcommand\Hyper[5]{{}_{#1}F_{#2} \left( \left.
    \begin{matrix}
      #3\\
      #4\\
    \end{matrix}
    \:\right|\: #5
    \right)}
\newcommand\qHyper[5]{{}_{#1}\phi_{#2} \left(
    \begin{matrix}
      #3\\
      #4\\
    \end{matrix}
    ; #5
    \right)}

\newcommand\comment[1]{\textcolor{blue}{\bf #1}}

%%%%%%%%%%%%%%%%%%%%%%%%%%%%%%%%%%%%%%%%%%%%%%%%%%%%%%%%%%%%%%%%

\title{Homework}


\begin{document}

\maketitle
\tableofcontents

\newpage
\section{Homework 1 (Due: Apr 5)}

\begin{problem}
 Let \(A\) be a nonempty set and let \(k\) be a positive integer with \(k \leq|A|\). The symmetric group \(S_A\) acts on the set \(B\) consisting of all subsets of \(A\) of cardinality \(k\) by 
 \[
   \sigma \cdot\left\{a_1, \ldots, a_k\right\}= \left\{\sigma\left(a_1\right), \ldots, \sigma\left(a_k\right)\right\}.
 \]
 \begin{enumerate}
 \item Prove that this is a group action.
 \item Describe explicitly how the elements (1 2) and (1 2 3) act on the six 2-element subsets of \(\{1,2,3,4\}\).
 \end{enumerate}

\end{problem}

\begin{problem}
  Let \(H\) be a group acting on a set \(A\). Prove that the relation
  \(\sim\) on \(A\) defined by \(a \sim b \) if and only if
  \( a=h b \) for some \(h \in H\) is an equivalence
  relation. (For each \(x \in A\) the equivalence class of \(x\) under
  \(\sim\) is called the orbit of \(x\) under the action of \(H\). The
  orbits under the action of \(H\) partition the set \(A\).)
\end{problem}


\begin{problem}
 In each of parts (1) to (5) give the number of nonisomorphic abelian groups of the specified order - do not list the groups:
\begin{enumerate}
\item order 100
\item order 576
\item order 1155
\item order 42875
\item order 2704
\end{enumerate}
\end{problem}

\begin{problem}
  In each of parts (1) to (5) give the lists of invariant factors for all abelian groups of the specified order:
\begin{enumerate}
\item order 270
\item order 9801
\item order 320
\item order 105
\item order 44100
\end{enumerate}
\end{problem}

\begin{problem}
  In each of parts (1) to (5) give the lists of elementary divisors
  for all abelian groups of the specified order and then match each
  list with the corresponding list of invariant factors found
  in the preceding problem:
\begin{enumerate}
\item order 270
\item order 9801
\item order 320
\item order 105
\item order 44100
\end{enumerate}
\end{problem}


\begin{problem}
  Let \(R\) be a ring with identity and let \(S\) be a subring of
  \(R\) containing the identity. Prove that if \(u\) is a unit in
  \(S\) then \(u\) is a unit in \(R\). Show by example that the
  converse is false.
\end{problem}

\begin{problem}
  Let \( R \) be a ring with \( 1\ne 0 \).
  \begin{enumerate}
  \item Prove that if \( a \) is a zero divisor, then it is not a
    unit.
  \item Prove that if \( ab=ac \) and \( a\ne0 \) is not a zero divisor,
  then \( b=c \).
  \end{enumerate}
\end{problem}

\begin{problem}
 Assume \(R\) is commutative. Prove that if \(P\) is a prime ideal of \(R\) and \(P\) contains no zero divisors then \(R\) is an integral domain. 
\end{problem}

\begin{problem}
  Let \( R \) be a ring with \( 1\ne 0 \).
  Let \(A=\left(a_1, a_2, \ldots, a_n\right)\) be a nonzero finitely
  generated ideal of \(R\). Prove that there is an ideal \(B\) which
  is maximal with respect to the property that it does not contain
  \(A\). [Use Zorn’s Lemma.]
\end{problem}

\begin{problem}
Let \(n_1, n_2, \ldots, n_k\) be integers which are relatively prime in pairs: \(\left(n_i, n_j\right)=1\) for all \(i \neq j\).
\begin{enumerate}
\item Show that the Chinese Remainder Theorem implies that for any \(a_1, \ldots, a_k \in \mathbb{Z}\) there is a solution \(x \in \mathbb{Z}\) to the simultaneous congruences
\[
x \equiv a_1 \bmod n_1, \quad x \equiv a_2 \bmod n_2, \quad \ldots, \quad x \equiv a_k \bmod n_k
\]
and that the solution \(x\) is unique \(\bmod n = n_1 n_2 \ldots n_k\).
\item Let \(n_i^{\prime}=n / n_i\) be the quotient of \(n\) by \(n_i\), which is relatively prime to \(n_i\) by assumption. Let \(t_i\) be the inverse of \(n_i^{\prime} \bmod n_i\). Prove that the solution \(x\) in (a) is given by
\[
x=a_1 t_1 n_1^{\prime}+a_2 t_2 n_2^{\prime}+\cdots+a_k t_k n_k^{\prime} \bmod n .
\]
Note that the elements \(t_i\) can be quickly found by the Euclidean Algorithm as described in Section 2 of the Preliminaries chapter (writing \(a n_i+b n_i^{\prime}=\left(n_i, n_i^{\prime}\right)=1\) gives \(t_i=b\) ) and that these then quickly give the solutions to the system of congruences above for any choice of \(a_1, a_2, \ldots, a_k\).

\item Solve the simultaneous system of congruences
\[
x \equiv 1 \bmod 8, \quad x \equiv 2 \bmod 25, \quad \text { and } x \equiv 3 \bmod 81
\]
and the simultaneous system
\[
y \equiv 5 \bmod 8, \quad y \equiv 12 \bmod 25, \quad \text { and } \quad y \equiv 47 \bmod 81
\]
\end{enumerate}
\end{problem}

\end{document}
