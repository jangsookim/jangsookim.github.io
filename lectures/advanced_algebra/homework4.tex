\documentclass{amsart}

\usepackage{amsmath,amsthm,amssymb,bm}
\usepackage{hyperref}
\usepackage{a4wide}
\usepackage{cleveref}
% \usepackage{refcheck}
\usepackage{graphicx,color}
\usepackage{tikz}
\numberwithin{equation}{section}

\newtheorem{thm}{Theorem}[section]
\newtheorem{lem}[thm]{Lemma}
\newtheorem{prop}[thm]{Proposition}
\newtheorem{cor}[thm]{Corollary}
\theoremstyle{definition}
\newtheorem{exam}[thm]{Example}
\newtheorem{defn}[thm]{Definition}
\newtheorem{conj}[thm]{Conjecture}
\newtheorem{question}[thm]{Question}
\newtheorem{problem}[thm]{Problem}
\newtheorem{remark}[thm]{Remark}
\newtheorem*{note}{Note}

% DO NOT DELETE THIS COMMENT!!! MACROS BELOW:
\newcommand\Gal{\operatorname{Gal}}
\newcommand\NI{\operatorname{NI}}
\newcommand\LRmin{\operatorname{LRmin}}
\newcommand\pts[1]{\textbf{[#1~points]}}
\newcommand\Pts[1]{\quad\textbf{[#1~points]}}
\newcommand\evencycle{\operatorname{evencycle}}
\newcommand\inv{\operatorname{inv}}
\newcommand\cycle{\operatorname{cycle}}
\newcommand\Motz{\operatorname{Motz}}
\newcommand\Fix{\operatorname{Fix}}
\newcommand\sgn{\operatorname{sgn}}
\newcommand\sym{\mathfrak{S}}
\newcommand\invol{\mathfrak{I}}
\newcommand\NN{\mathbb{N}}
\newcommand\QQ{\mathbb{Q}}
\newcommand{\CC}{\mathbb{C}}
\newcommand{\ZZ}{\mathbb{Z}}
\newcommand{\RR}{\mathbb{R}}
\newcommand\LL{\mathcal{L}}
\newcommand\FF{\mathbb{F}}
\newcommand\LH{\operatorname{LH}}
\newcommand\CH{\operatorname{CH}}

\newcommand\Mot{\operatorname{Mot}}
\newcommand{\Dyck}{\operatorname{Dyck}}

\newcommand\Par{\operatorname{Par}}
\newcommand\RPP{\operatorname{RPP}}
\newcommand\SSYT{\operatorname{SSYT}}
\newcommand\SYT{\operatorname{SYT}}

\newcommand\wt{\operatorname{wt}}

\renewcommand\vec[1]{\bm{#1}}
\newcommand\vx{\vec{x}}
\newcommand\vb{\vec{b}}
\newcommand\vla{\vec{\lambda}}
\newcommand\flr[1]{\left\lfloor #1\right\rfloor}
\newcommand\Qbinom[3]{\genfrac{[}{]}{0pt}{}{#1}{#2}_{#3}}
\newcommand\qbinom[2]{\Qbinom{#1}{#2}{q}}

\newcommand\hyper[5]{{}_{#1}F_{#2} \left(#3;#4;#5\right)}
\newcommand\qhyper[5]{{}_{#1}\phi_{#2} \left(#3;#4;#5\right)}
\newcommand\Hyper[5]{{}_{#1}F_{#2} \left( \left.
    \begin{matrix}
      #3\\
      #4\\
    \end{matrix}
    \:\right|\: #5
    \right)}
\newcommand\qHyper[5]{{}_{#1}\phi_{#2} \left(
    \begin{matrix}
      #3\\
      #4\\
    \end{matrix}
    ; #5
    \right)}

\newcommand\comment[1]{\textcolor{blue}{\bf #1}}

%%%%%%%%%%%%%%%%%%%%%%%%%%%%%%%%%%%%%%%%%%%%%%%%%%%%%%%%%%%%%%%%

\title{Homework}


\begin{document}

% \maketitle
% \tableofcontents


\section{Homework 4 (Due: May 31)}

\begin{problem}[Section 13.1, Exercise 2]
  Show that \(x^3-2 x-2\) is irreducible over \(\mathbb{Q}\) and let \(\theta\) be a root. Compute \((1+\theta)\left(1+\theta+\theta^2\right)\) and \(\frac{1+\theta}{1+\theta+\theta^2}\) in \(\mathbb{Q}(\theta)\).
\end{problem}


\begin{problem}[Section 13.2, Exercise 7]
  Prove that \(\mathbb{Q}(\sqrt{2}+\sqrt{3})=\mathbb{Q}(\sqrt{2}, \sqrt{3})\) [one inclusion is obvious, for the other consider \((\sqrt{2}+\sqrt{3})^2\), etc.]. Conclude that \([\mathbb{Q}(\sqrt{2}+\sqrt{3}): \mathbb{Q}]=4\). Find an irreducible polynomial satisfied by \(\sqrt{2}+\sqrt{3}\).
\end{problem}

\begin{problem}[Section 13.2, Exercise 19]
Let \(K\) be an extension of \(F\) of degree \(n\).
\begin{enumerate}
\item For any \(\alpha \in K\) prove that \(\alpha\) acting by left multiplication on \(K\) is an \(F\)-linear transformation of \(K\).
\item Prove that \(K\) is isomorphic to a subfield of the ring of \(n \times n\) matrices over \(F\), so the ring of \(n \times n\) matrices over \(F\) contains an isomorphic copy of every extension of \(F\) of degree \(\leq n\).
\end{enumerate}

\end{problem}

\begin{problem}[Section 13.2, Exercise 20]
Show that if the matrix of the linear transformation ``multiplication by \(\alpha\)'' considered in the previous exercise is \(A\) then \(\alpha\) is a root of the characteristic polynomial for \(A\). This gives an effective procedure for determining an equation of degree \(n\) satisfied by an element \(\alpha\) in an extension of \(F\) of degree \(n\). Use this procedure to obtain the monic polynomial of degree 3 satisfied by \(\sqrt[3]{2}\) and by \(1+\sqrt[3]{2}+\sqrt[3]{4}\)
\end{problem}




\begin{problem}[Section 13.4, Exercise 5]
  Let \(K\) be a finite extension of \(F\). Prove that \(K\) is a
  splitting field over \(F\) if and only if every irreducible
  polynomial in \(F[x]\) that has a root in \(K\) splits completely in
  \(K[x]\). [Use Theorems 8 and 27.]
\end{problem}


\begin{problem}[Section 13.4, Exercise 6]
Let \(K_1\) and \(K_2\) be finite extensions of \(F\) contained in the field \(K\), and assume both are splitting fields over \(F\).
\begin{enumerate}
\item Prove that their composite \(K_1 K_2\) is a splitting field over \(F\).
\item Prove that \(K_1 \cap K_2\) is a splitting field over \(F\). [Use the preceding exercise.]
\end{enumerate}
\end{problem}



\begin{problem}
  Let \( F \) be a field and let \( E,E' \) be algebraic closures of
  \( F \). Prove that there is an isomorphism \( \sigma:E\to E' \)
  such that \( \sigma|_F:F\to F \) is the identity map on \( F \).
\end{problem}


\begin{problem}[Section 13.5, Exercise 1]
  Prove that the derivative \(D_x\) of a polynomial satisfies \(D_x(f(x)+g(x))=D_x(f(x))+\) \(D_x(g(x))\) and \(D_x(f(x) g(x))=D_x(f(x)) g(x)+D_x(g(x)) f(x)\) for any two polynomials \(f(x)\) and \(g(x)\).
\end{problem}


\begin{problem}[Section 13.5, Exercise 7]
Suppose \(K\) is a field of characteristic \(p\) which is not a perfect field: \(K \neq K^p\). Prove there exist irreducible inseparable polynomials over \(K\). Conclude that there exist inseparable finite extensions of \(K\).
\end{problem}


\begin{problem}[Section 13.6, Exercise 6]
  Prove that for \( n \) odd, \( n>1 \),
  \( \Phi_{2 n}(x)=\Phi_n(-x) \).
\end{problem}

\end{document}
